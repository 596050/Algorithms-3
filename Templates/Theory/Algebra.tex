\Topic{Useful NTT mods and generators}
\begin{itemize}
\item $998244353 = 1 + 2^{23} \times 7 \times 17$, $g = 3$
\item $985661441 = 1 + 2^{22} \times 5 \times 47$, $g = 3$
\item $1012924417 = 1 + 2^{21} \times 483$, $g = 5$
\end{itemize}

\Topic{Fast Walsh–Hadamard transform} (Don't forget the normalization)
\begin{itemize}
  \item \textbf{XOR:} $T = \frac{1}{\sqrt{2}}\begin{bmatrix} 1 & 1 \\ 1 & -1 \end{bmatrix} = T^{-1}$
  \item \textbf{AND:} $T = \begin{bmatrix} 1 & 1 \\ 0 & 1 \end{bmatrix}, T^{-1} = \begin{bmatrix} 1 & -1 \\ 0 & 1 \end{bmatrix}$
  \item \textbf{OR:} $T = \begin{bmatrix} 1 & 0 \\ 1 & 1 \end{bmatrix}, T^{-1} = \begin{bmatrix} 1 & 0 \\ -1 & 1 \end{bmatrix}$
\end{itemize}

\Topic{Mod Inverse} $1$ to $n \bmod m$: inv$(i) = - \left\lfloor \frac{m}{i} \right\rfloor \cdot$ inv $(m \bmod i) \bmod m$

% \Topic{Zeta Transformation}
% \begin{itemize}
%   \item Disjoint Subset Convolution: $$ \sum_{c=0}^{N} \sum_{a=0}^{c} p_{c} CONV(p_{a}f * p_{c-a}g) $$
% \end{itemize}
% - To solve problem of type is it possible to divide a set into K sets, use the following dp[i][s] = 1 if it's possible to divide s into i sets else 0. Then find suitable relation between dp[i+1][s] and dp[i][.] \\
% - $ f[s] = \sum_{s1 U .. = s}{} a(s1) * b(s2) ... :: f = meu(zeta(a).*zeta(b).*.....) $ \\ - \# of perfect matching in a bipartile graph $ = \sum_{s \subseteq N} (-1)^{|N-s|} \prod_{i=1}^{N} \sum_{j \epsilon s} [ij \epsilon E]$ \\
% - \# of perfect matching in a general graph $ = \sum_{s\subseteq N} (-1)^{|N-s|} \binom{E[s]}{N/2}$ where $E[s] =$ # of edges that have end points in S. \\
% - to calcualte $f[s] = \sum_{s \subseteq s1} g[s1] $ use 
% $ f_{i+1}[s] =$ if $(i+1)^{th}$ bit is 1 $f_{i}[s]$ else $f_{i}[s] + f_{i}[s+2^{i}] $ \\
% - to calculate $V = \sum_{s1\cap s2 = \phi} f[s1]*g[s2]$ use $V=meu(supertsetsum(f) .* supersetsum(g))$ then $ans = V[Complete Set]$ \\ 

% \Topic{Correlation} $f[n]=\sum_{}{} h[n+x]*g[x]$ \\
% $temp=convolution(h,reverse(g))$ then $f[n]=temp[n+len(g)-1]$ \\
% - to find number of occurrences of string B in A where B can have '?' characters do : replace char c with $e^{\frac{2\pi i * (c -'a')}{26}}$ in A and replace char c with $e^{-\frac{2\pi i * (c -'a')}{26}}$, '?' with 0 in B. B occurs at i iff $correlation(A,B)[i] = |\# of B[i]'s \neq ?| $ \\

\Topic{2D Recurrence using FFT} For any 2D recurrence of the form $F_{n,p} = \sum_{i=0}^{k}a_{i}(n).F_{n-1,p-i}$. We can write it as follows:
$$\sum_{i=0}^{kn}F_{n,i}.x^{i} = \prod_{i=1}^{n} \sum_{j=0}^{k}a_{j}(i).x^{j}$$

\Topic{Trick for 2D FFT} $\frac{P(1)+P(\delta)+...+P(\delta^{n-1})}{n}$ is sum of all indexes of polynomial $P$ divisible by $n$ where $\delta = n^{th}$ root of unity.

\Topic{Leibniz formula for Determinants} $$det(A) = \sum_{\sigma \in S_{n}} \epsilon(\sigma) \prod_{i=1}^{n}a_{\sigma(i),i}$$ where $\epsilon$ is the levi-civita (1 for even, -1 for odd permutations)

\Topic{Lagrange Interpolation} For a polynomial of degree n, given $n+1$ distinct points $(x_i, y_i) \;\; \forall i \in [0, n]$
$$ y(x) = \sum_{i=0}^{n} y_i \prod_{j \ne i} \frac{x - x_j}{x_i - x_j} $$

\Topic{Functions of polynomials} To find the root of a functional $F(Q)$: Start with $Q_0 = c$, the correct constant term of the solution $P(x)$, and iterate. (each iteration doubles the number of terms)
$$ Q_{k+1} = Q_k - \frac{F(Q_k)}{F'(Q_k)} $$

\begin{itemize}
  \item Inverse: $F(Q) = QP - 1$, $Q_{k+1} = Q_k(2 - PQ_k)$
  \item Logarithm: $(\ln P(x))' = \frac{P'(x)}{P(x)}$, then integrate.
  \item Exponent: $F(Q) = ln(Q) - P$, $Q_{k+1} = Q_k(1 + P - ln Q_k)$
  \item $k$-th power: $P^k(x) = \exp[k \ln P(x)]$.
  Only works when $P(0) = 1$. Otherwise if $P(x) = \alpha x^t T(x)$ where $T(0) = 1$, then \\
  $P^k(x) = \alpha^k x^{kt} \exp[k \ln T(x)]$. (Note: also works for fractional k)
\end{itemize}